There are four modes to play \emph{Flat Hunt}: \emph{Hunt}, \emph{Escape}, \emph{Versus} and \emph{Demo}. Depending on the mode, zero (\emph{Versus}), one (\emph{Hunt/Escape}) or two (\emph{Demo}) parts are taken over by the computer.

  \begin{description}
    
    \item[Hunt]This is probably the most typical situation; the player tries to find the agent, which is played by the computer. Thus, the player only knows about every fifth move where the agent just was\ldots The agent shows himself only in rounds number 1, 3, 8, 13, 18, and 23. In these rounds, the exact route of the agent is displayed under \emph{History} in the status box at the bottom right corner and in the estate agent's own status box if opened. In all other rounds only the types of the transportation means he has taken so far are listed there.
    
    \item[Escape]This is the exact opposite of \emph{Hunt} mode: The agent is played by you, and the hunters are played by the computer. The hunters always move as close in your direction as possible, as they somehow manage to decode your transponder signal, and thus always know your precise location (so much for privacy\ldots). You just have to try to avoid them as long as possible\ldots
  
    \item[Versus]This is the multiplayer mode. One of the players is the agent; the other plays all the hunters. While the player of the agent is making a move, the player of the hunters is supposed to look away\ldots
   
    \item[Demo]This mode is more or less the opposite of the buzzword ``interactive'', but is about as entertaining as watching fish in an aquarium. The computer is playing against himself, trying to catch the agent as fast as possible.
  
  \end{description}
