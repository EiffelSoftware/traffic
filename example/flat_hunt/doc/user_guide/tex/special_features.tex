\subsection{Map Control}
Map control is fairly simple: you got two maps in a game scene, one of which only is a smaller version of the other one. The big map is on the left and that's where the action takes place, the little map on the right is meerely a navigation tool. To control the big map, use your mouse as follows:

  \begin{description}
    \item[left click:] only has an impact if clicked on a highlighted place
    \item[richt click + move mouse:] moves map in the direction of your mouse movement
    \item[middle click + move mouse up:] zoom in
    \item[middle click + move mouse down:] zoom out
  \end{description}

  When you \emph{left click + move mouse} in the little map, a light green rectangle is drawn between the point where your left mouse button is pressed down and the point when its released. As soon as you release the mouse button, the map segment that is inside this rectangle gets displayed on the big map.
 
\subsection{Music Player}
\emph{Flat Hunt} comes with an integrated music player and some default background music. Since not everyone likes the same sound, there is also the possibility to play your own.\\ 
Just put your \texttt{.ogg}-files in the directory \texttt{\$\{FLAT\_HUNT\}/resources/sound} before you start the Flat Hunt application. Flat Hunt will then automatically load all the \texttt{.ogg}-files from this directory and play them in alphabetical order (unless you enable shuffle, obviously). Music player control: see \ref{shortcuts} Keyboard Shortcuts.

\subsection{\label{shortcuts}Keyboard Shortcuts}

During the game, the following shortcuts are available:

  \begin{description}
    \item[p:] pause the game and show pause menu
    \item[s:] music player toggle shuffle
    \item[v:] music player decrease volume
    \item[shift + v:] music player increase volume (at startup the volume is already at maximum)
    \item[page up:] music player next song
    \item[page down:] music player previous song
    \item[q:] quit the application
  \end{description}
