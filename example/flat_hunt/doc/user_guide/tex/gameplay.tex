\emph{Playing Flat Hunt is not very difficult, especially for those that know the game ``Scotland Yard''\ldots}\\

\subsection{General Rules}

 The game lasts for at most 23 rounds. In these 23 rounds, the flat hunters try to find the estate agent, while he tries to avoid them (this is because he would rather rent the flats to elderly couples, since presumably they make fewer parties in the middle of the night\ldots).\\

In each round, every player can make one move on the public transport system. The estate agent is the first, then it's the hunters turn. One move is either 

\begin{itemize}
  \item  one or two stops by tram (colored lines),
  \item  one stop by train (thick orange lines),
  \item  or one stop by bus (thin light blue lines).
\end{itemize}

 A move with a certain transport can only be made if one has still enough tickets (see Figure 4), if there is a connection (obviously), and if there is no other player at that destination (and in the case of tram lines, if there is no hunter in between).\\

 \textcolor{red}{Attention: If you are at a bus-only stop, and you run out of bus tickets, you will get stuck there forever, so be careful\ldots}\\

The possible places you can move to are colored yellow (see Figure 5). To make a move, just click on one of those highlighted places. The red circle centers on the player whose turn it is, and in the status box at the right, the game status and information about the current player get displayed. If you want to know the status of another player just click on his picture at the bottom. Click again to close the just opened status box.\\

The game is over when

  \begin{description}
    \item[a)]the hunters could not find the estate agent within 23 rounds,
    \item[b)]one flat hunter moves onto the place where the agent currently is,
    \item[c)]or the hunters encircle the estate agent so that he cannot move anymore.
  \end{description}

In case a), the winner is the estate agent (he does not have to rent his flat to students), whereas in b) and c) it is the hunters that win, as they get to meet the estate agent on time and thus manage to find a flat.

\subsection{\label{game_modes}Game Modes}
There are four modes to play \emph{Flat Hunt}: \emph{Hunt}, \emph{Escape}, \emph{Versus} and \emph{Demo}. Depending on the mode, zero (\emph{Versus}), one (\emph{Hunt/Escape}) or two (\emph{Demo}) parts are taken over by the computer.

  \begin{description}
    
    \item[Hunt]This is probably the most typical situation; the player tries to find the agent, which is played by the computer. Thus, the player only knows about every fifth move where the agent just was\ldots The exact route of the agent can be seen at the bottom left (see Figure 6). The agent shows himself only in rounds number 1, 3, 8, 13, 18, and 23.
    
    \item[Escape]This is the exact opposite of \emph{Hunt} mode: The agent is played by you, and the hunters are played by the computer. The hunters always move as close in your direction as possible, as they somehow manage to decode your transponder signal, and thus always know your precise location (so much for privacy\ldots). You just have to try to avoid them as long as possible\ldots
  
    \item[Versus]This is the multiplayer mode. One of the players is the agent; the other plays all the hunters. While the player of the agent is making a move, the player of the hunters is supposed to look away\ldots
    
    \item[Demo]This mode is more or less the opposite of the buzzword ``interactive'', but is about as entertaining as watching fish in an aquarium. The computer is playing against himself, trying to catch the agent as fast as possible.
  \end{description}

  \subsection{Other}
  When you run \emph{Flat Hunt}, the first you'll see is a menu (see Figure 7). You can either let the default options in place and just select \emph{start game} or you can adjust the settings to your needs. \emph{Game mode} is explained in \ref{game_modes} Game Modes, \emph{number of hunters} and \emph{map size} should be self-explanatory and \emph{characters} specifies which pictures to use for the players. To toggle between the settings menu and the normal menu press \emph{tab}.\\

  During the game when you press \emph{p}, the pause menu is shown. \emph{Continue} makes the pause menu disappear and lets you resume the game, \emph{new game} takes you to the start menu and \emph{quit} quits the application.\\

  The game over menu is similar to the pause menu, only there is no \emph{continue} in this one. 

