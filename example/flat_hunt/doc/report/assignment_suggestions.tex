\paragraph{}
I tried to maintain most of the old exercises as well as give my own suggestions, which was quite difficult. But I hope some of them are usable nonetheless.

\subsection{Feature calls}
In class \texttt{START} there is a feature \textit{start} in which the students can place calls to make game settings and start the game.

\begin{lstlisting}
start is
		-- Adjust the game settings and start the game.
	do
--		POSSIBLE ASSIGNMENT (sample solution):
--		game.set_map ("./map/zurich_little.xml")
--		game.set_game_mode (Versus)
--		game.set_number_of_hunters (3)
		-- If `start_game' is left out, you will be stuck on the 
		-- start menu scene, because no game will be started, 
		-- even if you hit enter on "start game".
		start_game
	end
\end{lstlisting}

\subsection{Conditional Statements}
Give the students the class \texttt{PLAYER} but with

\begin{lstlisting}
enough_tickets (a_type: TRAFFIC_TYPE): BOOLEAN is
		-- Check if player has tickets to drive 
		-- with the transportation type `a_type'.
	require 
		a_type_valid: a_type /= Void and then is_valid_type (a_type)
	do
		if a_type.name @ (1) = 'b' then
			if bus_tickets > 0 then
				Result := True
			end
		elseif a_type.name @ (1) = 'r' then
			if rail_tickets > 0 then
				Result := True
			end	
		elseif a_type.name @ (1) = 't' then
			if tram_tickets > 0 then
				Result := True
			end		
		else
			Result := False
		end
	end
\end{lstlisting}
instead of the current \textit{enough\_tickets} and tell them to make this procedure a bit more readable and maybe hint at the inspect-when construct. Their goal would be to achieve the current \textit{enough\_tickets} routine (naturally this class should be changed before the students get it, otherwise it will be just a copy-paste exercise for them).

\subsection{Contracts}
Remove all contracts from some given class, and let the students fill them in.

\subsection{Loops}
In class \texttt{PLAYER\_DISPLAYER} you'll find the feature \textit{mark\_defeat} with an empty loop that the students get to fill (similar to the old exercise, but without animation, because an animation would indeed have been easily realizable with an \texttt{EM\_ANIMATABLE} but that would have eliminated the need for a loop and hence make the whole procedure pointless for this exercise).

\begin{lstlisting}
mark_defeat (a_surface: EM_SURFACE) is  
		-- Mark the defeat of the player.
	require
		a_surface_exists: a_surface /= Void
	local
		circle: EM_CIRCLE
		position: EM_VECTOR_2D
		count: INTEGER
	do
		-- Build `circle' at `position'.
		create position.make (player.location.position.x, 
		player.location.position.y)
		create circle.make_inside_box (picture.bounding_box)
		circle.set_line_color (white)
		circle.set_filled (False)
		circle.set_line_width (2)
		circle.draw (a_surface)

		-- POSSIBLE ASSIGNMENT
		-- With instructions for the students
		-- and sample solution:
		from
			-- Fill
--			count := 0
		until
			-- Replace `True' and fill.
--			count = 5
			True				
		loop
			-- Fill
--			circle.set_radius (circle.radius + 5)
--			circle.draw (a_surface)
--			count := count + 1
		end
	end  
\end{lstlisting}

\subsection{Inheritance}
A few suggestions:

\begin{itemize}
  \item Inherit from class \texttt{BRAIN}, redefine \textit{choose\_move} and implement an own artificial intelligence (for example a \texttt{DRUNKARD} like in last year's exercise). 
  \item Inherit from class \texttt{ESTATE\_AGENT\_DISPLAYER}, redefine \textit{draw} and for example implement an agent that shows himself in random rounds.
  \item Inherit from class \texttt{MENU}, effect \textit{set\_entry\_position} to customize the menu layout and redefine \textit{handle\_key\_down\_event} to change the menu's behavior.
\end{itemize}
The first option is the most suitable in my opinion (it was also used in last year's exercise), but I wanted to throw my other ideas in anyway\ldots

\subsection{Events}
Let the students study the class \texttt{BUTTON} or have them make class \texttt{TEXT\_BOX} clickable.

