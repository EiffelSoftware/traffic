For two years now, the department of Computer Science at ETH Zurich has been applying a new teaching approach called ``Inverted Curriculum'' \cite{bm03} \cite{mp03} \cite{mk04} to the \emph{Introduction to Programming} course. This technique is also referred to as a ``outside-in'' strategy of teaching. Rather than beginning with writing the infamous ``Hello World'' program, the students work from the start with a big software framework, which they gradually get to know better. At first, the exercises consist of merely calling a few library functions. Later on, control structures, Design by Contract, genericity and other advanced topics are introduced, some also using the framework.\\
The framework consists of a game-like application called \emph{Flat Hunt} and several libraries upon which the game is built. Namely \emph{TRAFFIC} \cite{sa05}, \emph{EiffelBase} and \emph{EiffelVision2}.\\

Due to the complexity of \emph{EiffelVision2} and the sheer unlimited multimedia capabilities of \emph{EiffelMedia} (which is the new name of \emph{ESDL} \cite{bb04} \cite{tgb03} \cite{rb05}), it was decided that \emph{Flat Hunt} should be redesigned to use \emph{EiffelMedia} for visualization rather than \emph{EiffelVision2}. And exactly that is the goal of this semester thesis. Well, actually almost exactly - for the goal also includes \emph{ESDL} extensions (see \autoref{esdl_ext}), suggestions for student assignments (see \autoref{assignments}) and last but not least also the redesign from a graphical point of view (meaning not only the application code will get a refresh but also the ``look and feel'' of the game).\\

Since \emph{Flat Hunt} is a teaching application, a great responsibilty lay upon my shoulders in producing a very clean design and code of impeccable quality (which is generally an utopia in software engineering, if you ask me - hence I don't claim to have completely accomplished that, but nonetheless endeavored to achieve). 
