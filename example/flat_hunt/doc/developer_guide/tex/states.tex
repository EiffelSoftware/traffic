\subsection{Overview}

Every game has at least two states: playing and game over. \emph{Flat Hunt} has six states in total; three playing states and three game over states (see Figure x). These game states are defined in class \texttt{GAME\_CONSTANTS}:

\texttt{Agent\_stuck, Agent\_stuck, Agent\_caught, Agent\_escapes, Prepare\_state, Play\_state, Move\_state: INTEGER is unique}

\subsection{Game Loop}
For each player in each round in \emph{Flat Hunt}, the game goes through the following states: \texttt{Prepare, Play} and \texttt{Move}. In addition, there are three game over states: \texttt{Agent\_stuck, Agent\_caught} and \texttt{Agent\_escaped}.

\begin{description}
    
  \item[Prepare] If the game is in this state, the current player gets a red circle and the possible moves are calculated and displayed. If the current player is the estate agent, and there are no possible moves, the agent is stuck and thus the game is over (state \texttt{Agent\_stuck}). If that is not the case, the game goes in state \texttt{Play}.
  
  \item[Play] In this state, if the current player is played by a human, the game waits until the human player clicks on one of the places that are highlighted. If the player is controlled by an artificial intelligence, then the best of the possible moves is calculated. The game then goes in state \texttt{Move}.
  
  \item[Move] In this state, the move selected in state \texttt{Play} is performed. After the move, the game checks if the player hits the place of the estate agent. If that is the case, the game goes into state \texttt{Agent\_caught}. If the agent did not get caught, and the round number is greater than 23, then the estate agent is the winner and the game goes into state \texttt{Agent\_escaped}. If none of the above is the case, then it's the next player's turn and the game loop starts again in state \texttt{Prepare}.

\end{description}
