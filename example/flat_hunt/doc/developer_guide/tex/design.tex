\emph{This and the following chapter should help you understand how the Flat Hunt software is structured. This chapter will give you an overview of the whole system organization and then provide an insight to some important classes.}

\subsection{Overview}

When opening \emph{Flat Hunt} in EiffelStudio, the cluster view in the bottom left corner of EiffelStudio shows many clusters. For you only the top-level clusters \emph{Traffic} and \emph{Flat\_hunt} are important.\\

To remove complexity, \emph{Flat Hunt} is structured in three clusters (see Figure 1): \emph{Model}, \emph{View} and \emph{Controller}. In each cluster, there are several classes, and sometimes there are subclusters.\\

\subsection{Controller cluster}

Cluster Controller is the fundamental cluster in \emph{Flat Hunt}. Here are the classes that ``control'' the actions. They make sure that the displayer classes in cluster View display the proper information, which they get from the Model classes. For example, feature prepare in class \texttt{GAME} controls the display update by calling \texttt{current\_player.displayer.display\_after\_move}.

\subsection{Model cluster}
In the cluster Model, there are two important parent classes: Class \texttt{PLAYER} and class \texttt{BRAIN}. \texttt{PLAYER} is the parent of \texttt{FLAT\_HUNTER} and \texttt{ESTATE\_AGENT}, and \texttt{BRAIN} is the parent of \texttt{HUMAN}, \texttt{FLAT\_HUNTER\_BOT} and \texttt{ESTATE\_AGENT\_BOT}. These Model classes describe the internal representation of ``real world'' objects. Here is a description of some of these classes.

\subsection{View cluster}
 This clusters job is to make sure that the user sees what is going on. It includes all scenes and menus, as well as displayers for the game players and status information.
